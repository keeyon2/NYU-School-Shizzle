\documentclass[ruled]{article}
\usepackage{relsize}

\begin{document}
\textbf{Keeyon Ebrahimi}\\
\textbf{NLP}\\
\textbf{HW7}\\ \\ \\

\textbf{Q1:} \\
Training with these five sentences, we see that we have to find the odds of \\
$np = n$ and $np = n\ pp$\\
\\
We also have to find the odds of $vp = v\ np\ pp$ and $vp = v\ np$\\ \\

We have a total of 15 $np$'s.  12 are $np = n$ and 3 are $np = n\ pp$.  This means that 
$$np = n \rightarrow 0.8$$  
\begin{center}
and 
\end{center} 
 $$np = n\ pp \rightarrow 0.2$$ 
\\ \\
For $vp$'s, we have 5 total $vp$'s.  2 are $v = v\ np\ pp$ and 3 are $v = v\ np$.  This means that
$$vp = v\ np\ pp \rightarrow 0.4$$
\begin{center}
and
\end{center}
$$vp = v\ np \rightarrow 0.6$$
\\ \\

Here are the probabilities of each of these 5 productions.

\begin{itemize}
\item[(a)]  (s (np (n Men) (pp (p of) (np (n distinction)))) (vp (v like) (np (n broccoli)))) 
\\
$$0.2 * 0.8 * 0.6 * 0.8 = 0.0768$$
\item[(b)]  (s (np (n Men)) (vp (v like)  (np (n ham)   (pp (p with) (np (n eggs))))))  
\\
$$0.8 * 0.6 * 0.2 * 0.8 = 0.0768$$
\item[(c)]  (s (np (n Men)) (vp (v serve) (np (n ham)   (pp (p with) (np (n eggs))))))  
\\
$$0.8 * 0.6 * 0.2 * 0.8 = 0.0768$$
\item[(d)]  (s (np (n Men)) (vp (v serve) (np (n eggs)) (pp (p with) (np (n gusto)))))  
\\
$$0.8 * 0.4 * 0.8 * 0.8 = 0.2048$$
\item[(e)]  (s (np (n Men)) (vp (v serve) (np (n eggs)) (pp (p to)   (np (n customers)))))  
\\
$$0.8 * 0.4 * 0.8 * 0.8 = 0.2048$$
\end{itemize}
\noindent\rule{15cm}{0.8pt}
\begin{center}
{\Large \textbf{Part 2}}
\end{center}
Now we must analyze: 
\\
\begin{center}
\textbf{“Delis serve pizza with relish.”}
\end{center}

Here are the two different parses for this sentence
\begin{verbatim}

1. (s (np (n Delis)) (vp (v serve) (np (n pizza)) (pp (p with) (np (n relish)))))


2. (s (np (n Delis)) (vp (v serve) (np (n pizza) (pp (p with) (np (n relish))))))

\end{verbatim}
Here are the different probabilities that would be assigned to the two different parses of \textbf{“Delis serve pizza with relish.”}\\  
\textit{Same numbering as the numbering right above}\\
\begin{itemize}
\item[1. ] $0.8 * 0.4 * 0.8 * 0.8 = 0.2048$
\item[2. ] $0.8 * 0.6 * 0.2 * 0.8 = 0.0768$
\end{itemize}
\begin{verbatim}


\end{verbatim}
\begin{center}
\textit{Problem 2 on next page}
\end{center}
\newpage

\textbf{Q2: } \\  \\
Now we will be conditioning the $vp$ on the head.

All the verb phrases have two possible heads, which are:
\textit{like} or \textit{serve}.
\\ \\ 
$$vp = v\ np\ pp \rightarrow (1.0\ chance\ serve)\ and\ (0.0\ chance\ like) $$ 
$$vp = v\ np \rightarrow (0.3333 \ chance\ serve)\ and\ (0.6666 \ chance\ like) $$ 
\\ \\
Now we must multiply the probability of each verb head to the $P(r(n))$ in each $vp$ calculation
\begin{itemize}
\item[(a)]  Delis serve pizza with relish
\\ \\
We know have to analyze each of the $vp$ possibilities with the head possibilities.

\begin{verbatim}

1. (s (np (n Delis)) (vp (v serve) (np (n pizza)) (pp (p with) (np (n relish)))))


2. (s (np (n Delis)) (vp (v serve) (np (n pizza) (pp (p with) (np (n relish))))))

\end{verbatim}
Here are the different probabilities that would be assigned to the two different parses of \textbf{“Delis serve pizza with relish.”}\\  
\textit{Same numbering as the numbering right above}\\
\begin{itemize}
\item[1. ] $0.8 * (0.4 * 1.0) * 0.8 * 0.8 = 0.2048$
\item[2. ] $0.8 * (0.6 *0.3333) * 0.2 * 0.8 = 0.02559744$
\end{itemize}

\item[(b)]  Men like pizza with relish
\begin{verbatim}

1. (s (np (n Men)) (vp (v like) (np (n pizza)) (pp (p with) (np (n relish)))))


2. (s (np (n Men)) (vp (v like) (np (n pizza) (pp (p with) (np (n relish))))))

\end{verbatim}
Here are the different probabilities that would be assigned to the two different parses of \textbf{“Men like	 pizza with relish.”}\\  
\begin{center}


\textit{Answer on next page}\\
\end{center}
\newpage
\textit{Same numbering as the numbering right above}\\
\begin{itemize}
\item[1. ] $0.8 * (0.4 * 0.0) * 0.8 * 0.8 = 0.0$
\item[2. ] $0.8 * (0.6 *0.6666) * 0.2 * 0.8 = 0.05119488$
\end{itemize}
\end{itemize}


\end{document}