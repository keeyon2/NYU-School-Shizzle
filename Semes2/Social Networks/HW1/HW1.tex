\documentclass[ruled]{article}
\usepackage{relsize}

\begin{document}
\textbf{Keeyon Ebrahimi}\\
\textbf{Social Networks}\\
\textbf{HW1}\\ \\ \\
\textbf{Problem 1}

\begin{itemize}
\item[Q1] 
\begin{verbatim}

3. When we think about a single aggregate measure to summarize
the distances between the nodes in a given graph, there are two
natural quantities that come to mind. One is the diameter, which
we define to be the maximum distance between any pair of nodes
in the graph. Another is the average distance, which Ñ as the term
suggests Ñ is the average distance over all pairs of nodes in the
graph. In many graphs, these two quantities are close to each other
in value. But there are graphs where they can be very different.
(a) Describe an example of a graph where the diameter is more
than three times as large as the average distance. (b) Describe how
you could extend your construction to produce graphs in which
the diameter exceeds the average distance by as large a factor as
youÕd like. (That is, for every number c, can you produce a graph
in which the diameter is more than c times as large as the average
distance?)


\end{verbatim}
\textbf{Solution: }\\
\begin{itemize}
\item[(a)]  An example of this graph would be a graph with a total of 6 nodes.  This means that we would have 5 different distances to node $X$.\\
We will label these 5 distances as $D_1$, $D_2$, $D_3$, $D_4$, $D_5$
\\
If the distances went like this, we would have the diameter be more than $3$ times more than the average distance to node $X$.

\textit{Distance to Node $X$}
\\
$D_1 = 1$ \\
$D_2 = 1$ \\
$D_3 = 1$ \\
$D_4 = 1$ \\
$D_5 = 60$ \\
\\

Now the average distance is $\frac{64}{5}$ and the diameter is $60$.  
\\ 
This fits problem a's condition because \\ \\ \\
{\LARGE $\frac{64}{5} * 3 < 60$ }
\\
\item[(b)]  Here is an equation to construct these graphs.  Lets label $n$ as the number of distances and $D_n$ will be the last distance, which we will assume is the diameter.  If this condition is true, we will have a graph that is larger than by at least a factor of $c$.
\\
\\
{\LARGE $\frac{D_n}{c} > \frac{D_1 + D_2 + D_3 + .... + D_n}{n}$ }
\\\\
\end{itemize}
\end{itemize}

\textbf{Problem 2}
\begin{itemize}
\item[Q2]
\begin{verbatim}

4. In the social network depicted in Figure 3.23 with each edge
labeled as either a strong or weak tie, which two nodes violate the
Strong Triadic Closure Property? Provide an explanation for your
answer.

\end{verbatim}
Node $B$ and node $E$ violate this structure.  Because Node $C$ has a strong tie to node $B$ and also, node $C$ has a strong tie to node to node $E$.  Because $B$ and $E$ are strongly tied to node $C$, yet they have no tie with each other, they violate the Strong Triadic Closure Property
\\ 
\begin{verbatim}

5. In the social network depicted in Figure 3.24, with each edge
labeled as either a strong or weak tie, which nodes satisfy the
Strong Triadic Closure Property from Chapter 3, and which do
not? Provide an explanation for your answer.

\end{verbatim}
Node $A$ and node $E$ violate the strong triadic closure property.  Node $C$ has a strong tie with Node $A$ and node $E$.  To follow the Strong Triadic Closure Property, $A$ and $E$ would then need to have a tie, but they do not in this figure.
\end{itemize}

\end{document}
