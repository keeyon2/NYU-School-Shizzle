\documentclass[ruled]{article}
\usepackage{relsize}

\begin{document}
\textbf{Keeyon Ebrahimi}\\
\textbf{Social Networks}\\
\textbf{HW3}\\ \\ \\
\textbf{Problem 1}

\begin{itemize}
\item[Q1] 
\begin{verbatim}

1. 
    (a) Show by direct calculation that the expected degree is np.
    (b) Where is the mode of the binomial distribution?
    (c) Compute directly the variance of the distribution

\end{verbatim}
\textbf{Solution: }\\
\begin{itemize}
\item[(a)] Expected Degree: \\

 {\Large $E(K) = \sum\limits_{k=1}^n k* \frac{n!}{k! (n - k)!} * p^k * (1-p)^{n - k}$   } \\ \\ \\
 {\Large $= \sum\limits_{k=1}^n \frac{n * (n-1)!}{(k-1)! (n - k)!} * p * p^{k-1} * (1-p)^{n - k}$   } \\ \\ \\
 {\Large $= np * \sum\limits_{k=1}^n k* \frac{(n-1)!}{(k-1)! (n - k)!} * p^{k-1} * (1-p)^{n - k}$   } \\ \\ \\

Use substitution with $a$ and $b$.  $a = k-1$ and $b = n-1$\\ \\

{\Large $E(K) = np * \sum\limits_{a=0}^b \frac{b!}{a! (b - a)!} * p^a * (1-p)^{b - a}$   } \\ \\ \\

{\Large We now see that \\ \\ $ \sum\limits_{a=0}^b \frac{b!}{a! (b - a)!} * p^a * (1-p)^{b - a}$ \\ \\ \\ is just another binomial distribution that will sum to 1, because probability of random variables sum to one.  Now we can deduce } 
\\ \\ \\ 

{\Large $E(K) = np * \sum\limits_{a=0}^b \frac{b!}{a! (b - a)!} * p^a * (1-p)^{b - a}$   } \\ \\ \\

{\Large $ = np * 1$   } \\ \\ \\
\textbf{Solution: }\\ \\
{\Large $ E(k) = np$} \\ \\ \\
\item[(b)]  Mode of the distribution:  We will find the mode of the Binomial distribution at \\ \\  {\Large $(n+1)p$ } \\ \\

\item[(c)]  Variance of the distribution:  \\ \\ \\ 
$Variance\ K = E(K^2) - (E(K))^2$
\\ \\ 
$(E(K))^2 = n^2 p^2$       We know this from part (a).
\\ 
\\
$E(K^2) = n^2 p^2 + np(1-p) $
\\ \\ \\
When we subtract the two, we get \\ \\
$Variance\ K = E(K^2) - (E(K))^2$ \\ \\
$ = n^2 p^2 + np(1-p) - n^2 p^2$ \\ \\
{\Large $ = np(1-p)$}\\ \\

\textbf{Solution: } \\ 
{\Large $ = np(1-p)$}\\ \\
\end{itemize}
\end{itemize}

\textbf{Problem 2}
\begin{itemize}
\item[Q2]
\begin{verbatim}

 In G(n, 1/n) what is the probability that there is a vertex of
degree log n? Give an exact formula; also derive simple approximations.

\end{verbatim}
The exact formula is: \\\\ \\ 
{\LARGE $ { n \choose \log(n)} (\frac{1}{n})^ {\log n} (1 - \frac{1}{n})^{n - \log n}$}
\\ \\ \\ \\ 

Here are a couple of examples.  If $n = 3$, we get 
\\ \\
{\LARGE $ { 3 \choose \log(3)} (\frac{1}{3})^ {\log 3} (1 - \frac{1}{3})^{3 - \log 3} \approx 0.4338765$}
\\ \\ \\
If $n = 5$, we get
\\ \\
{\LARGE $ { 5 \choose \log(5)} (\frac{1}{5})^ {\log 5} (1 - \frac{1}{5})^{5 - \log 5} \approx 0.2931279$} \\ \\ \\



\end{itemize}
\newpage
\textbf{Problem 3}
\begin{itemize}
\item[Q3]
\begin{verbatim}

 (a)  What is the expected number of triangles in G(n,d/n)
 (b)  What is the expected number of squares in G(n,d/n)
 (c)  What is the expected number of 4-cliques in G(n,d/n)

\end{verbatim}

\item[(a)] Probability that there is a cycle of 3, means that three nodes have to have a certain edge, which is a probability of $p^3$.  We know, $p = \frac{d}{n}$.  The total amount of possible node triangle possibilities is $n \choose 3$.  When we multiply these two, we get \\ \\
\textbf{Solution: }\\ \\
{\LARGE $(\frac{d}{n}) ^3 * {n \choose 3}$ }\\
\\ \\
\item[(b)] With the same logic as part a, we can deduce the expected number of squares with \\ \\
\textbf{Solution: } \\ \\
{\LARGE $(\frac{d}{n}) ^4 * {n \choose 4}$ }\\
\\ \\
\item[(c)]   Probability of 4 nodes being in a clique in this graph is \\ \\

{\LARGE $4 * (\frac{d}{n})^{3} $ }
\\ \\
Now the expected value of this means we have to see how many possible times this can happen, which we know is {\Large $n \choose 4$}.  To find our solution, we multiply these two together and we get \\ 

\textbf{Solution: } \\ \\
{\LARGE $4 * (\frac{d}{n})^{3} *{n \choose 4}$}
\end{itemize}

\end{document}
